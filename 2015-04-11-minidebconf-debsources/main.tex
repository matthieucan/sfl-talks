\documentclass{beamer}
\usepackage[utf8]{inputenc}
\usepackage{graphicx}
\usepackage{alltt}
\usepackage{listings}
\lstdefinestyle{sql}
               {language={SQL},
                 showstringspaces=false,
                 keywordstyle=\color{black}\bf,
               }
\usepackage{tikz}

% hack to get bold monospace:
\DeclareFontShape{OT1}{cmtt}{bx}{n}{<5><6><7><8><9><10><10.95><12><14.4><17.28><20.74><24.88>cmttb10}{}

\usetheme{Boadilla}
\usecolortheme{beaver}

\title{Debsources}

\subtitle{Dive into Debian source code!}

\author{\textbf{Matthieu Caneill}}

\institute[]{Debian contributor, PhD student at LIG}

\date[MiniDebconf, April 2015]{April 11, 2015\newline \alert{MiniDebconf} (Lyon, France)}

\begin{document}

\AtBeginSection[]
               {
                 \begin{frame}
                   \frametitle{Table of contents}
                   \tableofcontents[currentsection]
                 \end{frame}
               }
\newcommand{\background}{
\usebackgroundtemplate{
  \tikz\node[opacity=0.3]{
    \includegraphics[width=\paperwidth]{img/debsources.png}};}
}

\background

\begin{frame}[plain]
  \titlepage  
\end{frame}

\usebackgroundtemplate{}

\begin{frame}
  \frametitle{Acknowledgements}
  
  \begin{block}{Debsources}
    Initially developped at IRILL, by Stefano Zacchiroli and
    myself. Many persons have contributed patches since then.
  \end{block}
  \pause
  \begin{block}{Infrastructure}
    Debsources' servers are sponsored by IRILL.
  \end{block}
  \pause
  \begin{block}{Slides}
    Inspired by Stefano Zacchiroli's previous presentations.

    Licensed CC-BY-SA.
  \end{block}

\end{frame}

\begin{frame}
  \frametitle{Table of contents}
  \tableofcontents
\end{frame}

\section{Introduction}

\begin{frame}
  \frametitle{What is Debsources?}
  \begin{itemize}
    \item A web application to browse \alert{the source code} of
      Debian packages
    \item The infrastructure behind: \alert{database},
      \alert{plugins}, ...
  \end{itemize}
  \vspace{1cm}
  \pause
  \begin{block}{Play with it!}
    Navigate to \url{http://sources.debian.net}
  \end{block}
\end{frame}

\begin{frame}
  \frametitle{Home page}
  \makebox[\textwidth]{\includegraphics[width=\paperwidth]{img/screenshot-home.png}}
\end{frame}

\begin{frame}
  \frametitle{Source code display}
  \makebox[\textwidth]{\includegraphics[width=\paperwidth]{img/screenshot-file.png}}
\end{frame}

\begin{frame}
  \frametitle{So what?}
  \framesubtitle{Is this really useful?}
  ``I want to check the source code of \textit{cowsay}. What do?''
  \begin{block}{The old way}
    \texttt{cd /tmp\\
      apt-get source cowsay\\
      cd cowsay-3.03+dfsg1\\
      \ldots\\
      cd ..\\
      rm -r cowsay-3.03+dfsg1
    }
  \end{block}
  \small{Note that it only works with Debian-based systems, and not
    with your iPhone™.}
  \pause
  \begin{block}{The new way}
    \texttt{lynx http://sources.debian.net/src/cowsay/}
  \end{block}
  \small{Almost runs on your typewriter.}
\end{frame}

\section{Features}

\subsection{Debsources' features}

\begin{frame}
  \frametitle{Source code browsing}
  \begin{block}{Syntax highlighting}
    For all languages supported by \alert{highlight.js}: C, C++,
    Java, Python, Ruby, Makefile... and 112 others.
  \end{block}
  \pause
  \begin{block}{Included versions}
    Packages in\\

    \texttt{hamm, sink, potato, woody, sarge, etch, lenny, squeeze,
      \alert{wheezy}, \alert{jessie (testing)}, \alert{unstable},
      experimental, oldstable-updates, stable-updates,
      proposed-updates, testing-proposed-updates, wheezy-backports,
      squeeze-lts}\\

    are in Debsources.
  \end{block}
    
\end{frame}

\begin{frame}
  \frametitle{Searching}
  You can search for:
  \begin{itemize}
  \item Packages
  \item Files
  \item File content
    \begin{itemize}
    \item ctags
    \item regular expressions through \texttt{codesearch.debian.net}
    \end{itemize}
  \end{itemize}
  \pause
  \begin{block}{Content indexing}
    Searching is fast, thanks to PostgreSQL's indexes.
  \end{block}
\end{frame}

\begin{frame}
  \frametitle{Advanced search}
  \makebox[\textwidth]{\includegraphics[height=0.8\paperheight]{img/screenshot-search.png}}
\end{frame}

\begin{frame}
  \frametitle{Annotations and highlighting}
  \begin{block}{Suppose...}
    \begin{itemize}
      \item I'm a \alert{developer}: \textit{I want to share a
        precise location in the source code of package X.}
      \item I'm a \alert{user}: \textit{I can't compile software X,
        it fails at line 42 in the file Y.}
      \item I'm a \alert{static source code analyzer}: \textit{I
        found a semantic error in file Y.}
    \end{itemize}
  \end{block}
\end{frame}

\begin{frame}
  \frametitle{Annotations and highlighting}
  \texttt{http://sources.debian.net/src/\\
    \textbf{package}/\textbf{version}/\textbf{path/to/file.c}?hl=\textbf{a:b}\&msg=\textbf{a:b:c}\#L\textbf{XX}}
  \vspace{1cm}
  \pause
  \begin{columns}
    \column{.4\textwidth}
    \begin{tabular}{ l l }
      package: & cowsay\\
      version: & 3.03-3\\
      path: & cowsay\\
      highlight: & 32:36\\
      message: & 30:Debian:rocks
      %\\line: & 30
    \end{tabular}
    \pause
    \column{.6\textwidth}
    \includegraphics[width=7.5cm]{img/screenshot-hl.png}
  \end{columns}
\end{frame}

\begin{frame}
  \frametitle{Annotations and highlighting}
  \framesubtitle{\alert{Developer}: I want to share a precise
    location in the source code of package X.}
  \makebox[\textwidth]{\includegraphics[width=\paperwidth]{img/screenshot-hl1.png}}
\end{frame}

\begin{frame}
  \frametitle{Annotations and highlighting}
  \framesubtitle{\alert{User}: I can't compile software X, it fails
    at line 42 in the file Y.}
  \makebox[\textwidth]{\includegraphics[width=\paperwidth]{img/screenshot-hl2.png}}
\end{frame}

\begin{frame}
  \frametitle{Annotations and highlighting}
  \framesubtitle{\alert{Static analyzer}: I found a semantic error in
    file Y.}
  \makebox[\textwidth]{\includegraphics[width=\paperwidth]{img/screenshot-hl3.png}}
\end{frame}

\begin{frame}
  \frametitle{Duplicated files}
  All the files are in the database, along with their checksum.

  The duplicates can be computed, for every file.

  \vspace{1cm}
  \makebox[\textwidth]{\includegraphics[width=0.9\paperwidth]{img/screenshot-duplicates1.png}}
\end{frame}

\begin{frame}
  \frametitle{Duplicated files}
  \makebox[\textwidth]{\includegraphics[width=0.9\paperwidth]{img/screenshot-duplicates2.png}}
\end{frame}

\begin{frame}
  \frametitle{Integration in the ecosystem}
  \begin{block}{Codesearch}
    \url{http://codesearch.debian.net} is used for regular expression
    searches, and redirects its results back to Debsources.

    Credits: Michael Stapelberg
  \end{block}
  \pause
  \begin{block}{Package tracking systems}
    The old (\url{http://packages.qa.debian.org}) and new
    (\url{http://tracker.debian.org}) PTS provide links to Debsources
    (``browse source code'').

    Credits: Paul Wise
  \end{block}
  \pause
  \begin{block}{Need to embed code somewhere?}
    $<$iframe$>$s embedding of files content is supported (see
    documentation).
  \end{block}
\end{frame}

\begin{frame}
  \frametitle{Statistics}
  \begin{columns}
    \column{.6\textwidth}
    \begin{itemize}
    \item Code source metrics for every package.
    \item Plugins: size, ctags, sloccount
    \end{itemize}
    \pause
    
    \column{.4\textwidth}
    \includegraphics[height=6cm]{img/screenshot-stats.png}
  \end{columns}
\end{frame}

\begin{frame}
  \frametitle{Statistics}
  Aggregated statistics are available at
  \url{http://sources.debian.net/stats/}.
  \begin{block}{Metrics}
    \begin{itemize}
    \item Disk usage
    \item SLOC (source lines of code)
    \item Number of source packages
    \item Number of files
    \item Number of ctags (symbols)
    \end{itemize}
  \end{block}
\end{frame}

\begin{frame}
  \frametitle{Statistics}
  Currently in the \textit{unstable} suite:
  ~\\ ~\\
  \begin{tabular}{ l | l }
    \alert{Disk usage} & $\simeq$ 210 GB\\
    \alert{SLOC} (source lines of code) & 1,002,329,283\\
    Number of \alert{source packages} & 22,763\\
    Number of \alert{files} & 10,565,532\\
    Number of \alert{ctags} (symbols) & 114,732,392\\
  \end{tabular}
\end{frame}

\begin{frame}
  \frametitle{Statistics}
  \framesubtitle{Fancy graphs}
  \makebox[\textwidth]{\includegraphics[width=0.7\paperwidth]{img/sloc_bar_plot.png}}
\end{frame}
  

\begin{frame}
  \frametitle{API}
  Along the graphical web interface, an API provides the same
  functionalities.
  \pause
  \begin{block}{Examples}
    \texttt{\textbf{curl http://sources.debian.net/api/ping/}\\
      \{
        ``status'': ``ok'',\\
        ~~``http\_status\_code'': 200,\\
        ~~``last\_update'': ``Fri, 10 Apr 2015 10:16:31 -0000''
      \}}
    \\ ~ \\
    \texttt{\textbf{curl
        http://s.d.n/api/info/package/cowsay/3.03-3/}\\
      \{\\
        ``pkg\_infos'': \{\\
        ~~  ``suites'': [\\
        ~~~~      ``woody''\\
      ...}
  \end{block}
  Documentation at \url{http://sources.debian.net/doc/api/}.
\end{frame}

\subsection{What's new?}

\begin{frame}
  \frametitle{What's new?}
  \begin{itemize}
  \item Many new features since DebConf14
    \pause
  \item OPW student: Jingjie Jiang (sophiejjj)
    \pause
  \item Many new contributors:
    \begin{itemize}
      \begin{columns}
        \column{.3\textwidth}
    \item Stefano Zacchiroli: 688
    \item Matthieu Caneill: 455
    \item sophiejjj: 58
    \item Orestis Ioannou: 10
    \item Jason Pleau: 10
    \item Akshita Jha: 7
    \item Clément Schreiner: 6
      \column{.3\textwidth}
    \item Jingjie Jiang: 5
    \item Luciano Bello: 1
    \item Paul Wise: 1
    \item tessy joseph: 1
    \item Christophe Siraut: 1
    \item sodamatt: 1
    \item Tapasweni Pathak: 1
      \end{columns}
    \end{itemize}
  \end{itemize}
\end{frame}

\begin{frame}
  \frametitle{What's new?}
  \framesubtitle{Multiple pop-up messages}
  Credits: Orestis Ioannou and Jason Pleau.
  \vspace{7mm}
  ~\\
  \makebox[\textwidth]{\includegraphics[width=0.9\paperwidth]{img/screenshot-doublepopup.png}}
\end{frame}

\begin{frame}
  \frametitle{What's new?}
  \framesubtitle{Blueprints support}
  Credits: Jingjie Jiang
  \vspace{7mm}
  ~\\  
  \begin{block}{Blueprints}
    \begin{itemize}
    \item Flask apps embedded and plugged together
    \item Implied a big refactoring
    \item New features incoming
    \end{itemize}
  \end{block}
\end{frame}

\begin{frame}
  \frametitle{What's new?}
  \framesubtitle{Detailed directory listing}
  Credits: Jingjie Jiang
  
  ~\\
  \makebox[\textwidth]{\includegraphics[width=0.6\paperwidth]{img/screenshot-ls.png}}
\end{frame}

\begin{frame}
  \frametitle{What's new?}
  \framesubtitle{File edition in-browser}
  Credits: Raphael Geissert
  
  ~\\
  \begin{block}{File edition}
    A plugin for Iceweasel and Chromium enables the edition of files in
    your browser.

    A patch ready-to-be-sent™ is generated!
  \end{block}
\end{frame}

\begin{frame}
  \frametitle{What's new?}
  \framesubtitle{And many many other features...}
  \begin{itemize}
  \item Refactoring
    \pause
    \begin{itemize}
    \item Debsources as a top-level Python module
      \pause
    \item Configuration loader
      \pause
    \item Flake8 compliance (Zack, Jingjie Jiang, and others)
      \pause
    \end{itemize}
  \item Test coverage (Jingjie Jiang, Clément Schreiner, and others)
    \pause
  \item Case-insensitive package name search (Akshita Jha)
    \pause
  \item ?lang=LANG to override detected language (Jingjie Jiang)
    \pause
  \item Symbolic links handling (Jingjie Jiang)
    \pause
  \item Better statistics charts (Orestis Ioannou)
    \pause
  \item Python3 support (Zack)
  \end{itemize}
\end{frame}

\subsection{Roadmap}

\begin{frame}
  \frametitle{Roadmap}
  \begin{block}{Static analysis}
    \begin{itemize}
      \item Automatic runs of static analysis tools (e.g. clang,
        coccinelle) on all Debian packages
      \item Statistics gathering on bugs evolution
      \item $\to$ Debile, Firewoes
    \end{itemize}
  \end{block}
  \pause
  \begin{block}{copyright.debian.net}
    \begin{itemize}
    \item Web application
    \item What is the license of package X? Is it compatible with
      package Y?
    \item Licenses searching/browsing
    \item Statistics gathering
    \end{itemize}
  \end{block}
\end{frame}

\begin{frame}
  \frametitle{Roadmap}
  \framesubtitle{And many smaller items}
  \begin{itemize}
  \item more live stats
    \pause
  \item file name search
    \pause
  \item binary package $\to$ source package redirection
    \pause
  \item tarball-in-tarball support
    \pause
  \item 100\% test suite coverage
    \pause
  \item file-level deduplication
    \begin{itemize}
    \item \lstinline[style=sql]{select count(*) from checksums;}
      \hfill $\rightarrow$ 35'370'653
    \item
      \lstinline[style=sql]{select count(distinct sha256) from
        checksums;}
      \hfill $\rightarrow$ 15'822'745
    \item[] \hfill $\Rightarrow$ \alert{deduplicated core:
      $\approx$ 45\%}
    \end{itemize}
  \end{itemize}
\end{frame}

\section{Technologies}

\begin{frame}
  \frametitle{Technologies}
  \framesubtitle{What languages and technologies do we use?}
  \begin{itemize}
  \item \textbf{Code base}: (almost) entirely in Python
    \pause
  \item \textbf{Web application}: Flask framework, Jinja2 templates,
    HTML/CSS/Javascript
    \pause
  \item \textbf{Database}: PostgreSQL
    \pause
  \item Apache web server, SQLAlchemy, ...
  \end{itemize}
\end{frame}

\begin{frame}
  \frametitle{Overview}
  \framesubtitle{Architecture}
  \begin{center}
    \includegraphics[width=0.9\textwidth]{img/architecture}
  \end{center}
\end{frame}

\begin{frame}
  \frametitle{Overview}
%  \framesubtitle{Database and plugins}
  ~\hfill {\small data model (excerpt)}
  \vspace{-0.5cm}
  \begin{center}
    \includegraphics[width=\textwidth]{img/dbschema-simpl2}
  \end{center}
  \vspace{-2.8cm}
  \begin{center}
    \footnotesize
    ~\hspace{6.5cm}
    \begin{tabular}{@{}l|r@{}}
      \multicolumn{1}{c|}{\textbf{table}}
      & \multicolumn{1}{c}{\textbf{rows}\footnote{snapshot, 31
          July 2014}} \\
      \hline
      suites\_info   &          16 \\
      \hline
      package\_names &      29,286 \\
      (source) packages &   83,597 \\
      suites (mapping) &   120,550 \\
      \hline
      metrics (e.g., \texttt{du})
      &      83,597 \\
      sloccounts    &     298,360 \\
      checksums     &  35,370,653 \\
      ctags         & 358,773,259 \\
    \end{tabular}
  \end{center}
\end{frame}

\begin{frame}{Disk usage}
  \begin{columns}
    \begin{column}{0.5\textwidth}
      \begin{itemize}
      \item unpacked sources: \hfill 609 GB
        \note[item]{this is without any deduplication; I'll
          get back to that}
      \item PostgreSQL DB: \hfill 111 GB
      \item Source mirror: \hfill 71 GB
      \end{itemize}
      \alert{Hosting requirements}: \hfill $\approx$ 800
      GB\\[1ex]
      ~\hfill {\footnotesize (31 July 2014)}
    \end{column}
    \pause
    \begin{column}{0.5\textwidth}
      \begin{figure}
        \centering
        \includegraphics[width=\textwidth]{img/disk-usage}
        \caption{unpacked sources trend (peek due to
          \texttt{archive.d.o}
          injection)}
      \end{figure}
    \end{column}
  \end{columns}
\end{frame}

\section{Research platform}

\begin{frame}
  \frametitle{Research platform}
  \begin{block}{Facts}
    \begin{itemize}
    \item Debsources is a \alert{huge} software collection.
    \item Homogeneous: all software follow Debian's packaging format.
    \item It is up-to-date.
    \end{itemize}
  \end{block}
  \pause
  
  \begin{block}{Software evolution}
    \begin{itemize}
    \item 20 years of source code evolution.
    \item Plugins to compute stats.
    \end{itemize}
  \end{block}
  \pause
  
  \alert{Nice charts can be computed with this!}

  Example: What are the trending programming languages?
\end{frame}

\begin{frame}
  \frametitle{Research platform}
  \framesubtitle{Software metrics evolution over Debian releases}
  \vspace{-0.3cm}
  \begin{figure}
    \centering
    \includegraphics[width=0.85\textwidth]{img/size-combined}
    %\caption{}
  \end{figure}
\end{frame}

\begin{frame}
  \frametitle{Research platform}
  \framesubtitle{File size per language, evolution over Debian releases}
  \vspace{-0.3cm}
  \begin{figure}
    \centering
    \includegraphics[width=0.85\textwidth]{img/size-file-per-language}
    %\caption{}
  \end{figure}
\end{frame}

\begin{frame}
  \frametitle{Research platform}
  \framesubtitle{Absolute evolution of SLOC per language, over Debian releases}
  \vspace{-0.3cm}
  \begin{figure}
    \centering
    \includegraphics[width=0.85\textwidth]{img/sloc-evol-abs}
    %\caption{}
  \end{figure}
\end{frame}

\begin{frame}
  \frametitle{Research platform}
  \framesubtitle{Relative evolution of SLOC per language, over Debian releases}
  \vspace{-0.3cm}
  \begin{figure}
    \centering
    \includegraphics[width=0.85\textwidth]{img/sloc-evol-rel}
    %\caption{}
  \end{figure}
\end{frame}

\begin{frame}
  \frametitle{Research platform}
  \framesubtitle{Publications}
  \begin{itemize}
    \item Matthieu Caneill, Stefano Zacchiroli. \textbf{Debsources:
      Live and Historical Views on Macro-Level Software Evolution.} In
      proceedings of ESEM 2014: 8th International Symposium on
      Empirical Software Engineering and Measurement.
    \item Stefano Zacchiroli. \textbf{The Debsources Dataset: Two
      Decades of Debian Source Code Metadata.} To appear in
      proceedings of MSR 2015: The 12th Working Conference on Mining
      Software Repositories.
  \end{itemize}
  \vspace{1cm}
  \small{You can find the PDFs of the articles on \url{http://sources.debian.net/doc/}.}
\end{frame}

\section{Hacking}

\begin{frame}
  \frametitle{Hacking}
  \framesubtitle{How can I contribute?}
  \begin{block}{Step 1: clone Debsources git repository}
    \texttt{git clone git://anonscm.debian.org/qa/debsources.git}
  \end{block}
  \pause
  \begin{block}{Step 2: Set-up a development environment}
    \begin{itemize}
    \item Follow the instructions in the \alert{HACKING} file,
    \item Or \texttt{docker run}!
      \url{https://github.com/matthieucan/Dockerfiles/debsources}\\ Will
      setup a \alert{Docker container} with all batteries included:
      dependencies, database, test data, configuration.
    \end{itemize}
  \end{block}
  \pause
  \begin{block}{Step 3: open your editor and hack!}
    \alert{Bugs list}:
    \url{https://bugs.debian.org/cgi-bin/pkgreport.cgi?pkg=qa.debian.org;tag=debsources}
    \\
    or: implement your own \alert{plugin} (see examples), add \alert{features}, etc.
  \end{block}
\end{frame}

\background

\begin{frame}[plain]
  \center{
    {\huge Thanks!}
    {\huge Questions?}\\
    \vspace{1.5cm}
    {\Large Matthieu Caneill}\\
    \url{http://matthieu.io}\\
    \vspace{1.5cm}
    Slides: derived from Stefano Zacchiroli
    \\{\small \url{http://upsilon.cc/~zack/talks/2014/20140826-dc14-debsources.pdf}}
    \\Licensed CC-BY-SA.
  }
\end{frame}

\end{document}